\documentclass{article}
\usepackage[a4paper,left=10mm,right=10mm,top=15mm,bottom=15mm]{geometry}
\usepackage{amssymb,amsthm,latexsym,amsfonts, amsmath, bm}
\usepackage{enumerate}
\usepackage{extarrows}
\usepackage{tikz}
\usepackage{caption}  
\usepackage{graphicx}
\usepackage{float}
\usepackage[colorlinks,linkcolor=black]{hyperref}
\usetikzlibrary{backgrounds}
\usetikzlibrary{trees,positioning,arrows}
\usepackage{german}
\title{Übungen zur Algorithmischen Bioinformatik I\\
Hausarbeit 3\\
Komplexität schwerer Probleme und Entscheidbarkeit}
\author{Xiheng He }
\date{Juni 2021}
\linespread{1.5}
\begin{document}
\maketitle
\begin{flushleft}
\textbf{1. Aufgabe: Das Traveling Salesperson Problem (TSP)}
\newline
\begin{enumerate}[(a)]
\item Schreiben Sie ein Programm, das die optimale Tour unter allen möglichen Permutationen bestimmt. 
Testen Sie das Programm fur die Instanz des TSP aus der Vorlesung. 
\newline 
Usage: -mode [TSP|EPP|NN|MST]
\item \textbf{Hamilton vs Euler:} Was macht das TSP bzw. HC Problem schwer? Begründen Sie! Grenzen
Sie es dazu vom Euler-Kreis-Problem (EKP) ab. Wie sieht es fur EKP hinsichtlich Entscheidbarkeit
und Komplexität aus?
\newline
Das HC (Hamilton Kreis) Problem ist zu entscheiden, ob ein solcher Kreis (geschlossener Pfad) in einem gegebenen 
Graph existiert, der jeden Knoten genau einmal besucht. 
\newline
Das EKP (Euler-Kreis-Problem) ist zu bestimmen, 
ob ein solcher Kreis (geschlossener Pfad) in einem gegebenen Graph existiert, der jede genau einmal enthält.
\newline
Das TSP (Traveling Salesperson Problem) besteht darin, eine Reihenfolge für den Besuch mehrerer Orte so zu wählen, 
dass keine Station außer der ersten mehr als einmal besucht wird, die gesamte Reisestrecke des Handlungsreisenden 
möglichst kurz und die erste Station gleich der letzten Station ist. TSP kann auch so betrachtet werden, ob einen 
Hamilton Kreis mit Länge $\leq w$  in einem gegebenen Graph existiert da HC auf TSP leicht reduziert werden kann.
\newline
Im Einscheidungsproblem HC wollen wir also feststellen, ob ein gegebener Graph einen einfahcne Kreis auf allen Knoten
als Teilgraphen enthält. Deshalb müssen wir alle alle Permutationen der Vertices betrachten die insgesamt $|V|!$ sind.
Die schwierigkeit liegt daran, dass kein Algorithmus existiert, der das Problem in polynomieller Laufzeit löst. 
\newline
Aus \textbf{Theorem 5.26} Skript S.257: HC ist $\mathcal{N}\mathcal{P}-vollständig$.
\newline
Da wir einen Hamilton Kreis mit höchstens $w$ Gewicht in einem gegebenen Graph bestimmen muss um das TSP zu lösen,
zeigt dies, dass TSP mindestens so schwierig ist wie das HC Problem. Deshalb existiert auch kein polynomieller Algorithmus
für TSP.
\newline
Für das EKP (Euler-Kreis-Problem) kann es aber durch Satz von Euler-Hierholze in polynomieller Zeit bestimmt werden, 
ob ein Euler-Kreis in einem gegebenen Graph existiert.
\newline
Satz von Euler:
\begin{itemize}
    \item Sei $G$ ein ungerichteter Graph dann sind folgende Aussagen äquivalent:
    \begin{itemize}
        \item $G$ ist eulersch
        \item jeder Knoten in $G$ hat geraden Grad
    \end{itemize}
    \item Sei $G$ ein gerichteter Graph dann sind folgende Aussagen äquivalent:
    \begin{itemize}
        \item $G$ ist eulersch
        \item für jeden Knoten in $G$ sind Eingangsgrad und Ausgangsgrad gleich
    \end{itemize}
\end{itemize}
Somit ist das Euler-Kreis-Problem entscheidbar und gehört zur Klasse $\mathcal{P}$. 
\item Schreiben Sie ein Programm, das einen bzw. die Eulerkreise in einem gewichteten Graphen bestimmt. 
Testen Sie das Programm fur die Instanz des TSP aus der Vorlesung und dem Königsberg-Graphen.
\item Wenden Sie Ihre Programme (mit Bedacht!) auf realistische Probleminstanzen (siehe Aufgabe
4) an. Definieren Sie ggfs. (wenn die bereitgestellten Instanzen zu schwierig/groß sind) geeignete
Teilprobleme, um die praktische Anwendbarkeit Ihrer Implementierungen zu untersuchen.
\end{enumerate}
\textbf{2. Aufgabe: Approximationen des TSP}
\newline
Implementieren Sie die (einfachen) Approximationen aus der Vorlesung, die auf nächsten Nachbarn
(NN, Greedy) und minimalen Spannbäumen (MST) aufbauen. ``Nearest Neighbor'' reist immer
von der aktuellen Stadt zu nächstgelegenen noch nicht besuchten Stadt. ``Greedy'' fügt immer die
kürzeste noch verfügbare Kante hinzu, die zwei Teilpfade verbindet (also Kurzzyklen vermeidet!).
MST meint die ``twice-around-the-spanning-tree'' Approximation aus der Vorlesung.
Können Sie Instanzen angeben, die für die jeweiligen Approximationen ungünstig sind? Sind die
angegebenen Approximationgüten scharf (d.h. gibt es Instanzen, die nicht zu besseren Lösungen
führen wie die Güteschranke)?
\newline
Usage: -mode [TSP|EPP|NN|MST]
Greedy sucht immer nur den kürzesten Pfad. Eine ungünstige Instanz:
\newline
\begin{figure}[H]
    \centering
    \includegraphics[width=0.2\textwidth]{tsp_greedy.png}
    \caption{Ungünstige Instanz für Greedy}
\end{figure}
Optimale Tour des TSP ist a, c, d, b, a wobei Der Greedy Algorithmus a, b, c, d, a ausgibt.
\newline
Die minimale Spannbäume garantieren nicht die Erzeugung des Hamilton-Rings mit dem kleinsten Gewicht.
Und es ist nicht unbedingt, dass die Distanz der jeden Schritt in MST die minimale Distanz im euklidischen Raum ist (d.h. erfüllt die Dreiecksungleichung ).
Ein gute Bespiele dafür ist die Instanz der TSP aus Vorlesung. 
\newline
\textbf{3. Aufgabe: ILP Lösung des TSP}
\begin{itemize}
    \item Diese Ungleichungen definieren gültige TSP Touren (das TSP-Polytop) allerdings werden dafür sehr viele
    Ungleichungen gebraucht. Wie viele?
    \newline
    Für ``degree constraints'' muss für jede Knoten eine Ungleichung paarweise erstellt werden damit ergibt sich insgesamt $n(n - 1)/2$ Ungleichungen.
    \newline
    Für ``sub-tour elimination constraints'' muss $n - 1$ Ungleichungen je ein Knotenpaar erstellt werden damit ergibt sich insgesamt $n(n - 1)^2 / 2$ Ungleichungen.
    \item Sie  verhindern dann auch Kurzzyklen. Wieso?
    \newline
    Gegeben sei die Ungleichung für die Sub-Tour Elimination:
    \begin{equation}
        t_i - t_j + n \cdot x_{ij} \leq n - 1
    \end{equation}
    Aus (1) ergibt sich:
    \begin{equation}
        t_i - t_j + n \cdot x_{ij} \leq n - 1 \Longrightarrow t_i - t_j + 1 - n(1 - x_{ij}) \leq 0
    \end{equation}
    \begin{equation}
        \forall x_{ij} = 0, t_i \leq n \land t_j \geq 0 \Longrightarrow t_i - t_j \leq n - 1
    \end{equation}
    \begin{equation}
        \forall x_{ij} = 1, t_i - t_j + 1 - n(1 - x_{ij}) \leq 0 \Longrightarrow t_i - t_j + 1 \leq 0
    \end{equation}
    (3) bedeutet, dass die Sub-Tour Elimination Einschränkung erfüllt wird wenn $x_{ij} = 0$ nämlich Knoten $i,j$ keinen Pfad besitzen.
    \newline
    (4) bedeutet, dass die Sub-Tour Elimination Einschränkung erfüllt wird auch wenn $x_{ij} = 1$ nämlich Knoten $i,j$ einen Pfad besitzen.
    \newline
    Betrachten wir drei Knoten $i, j, k$. Falls eine Subtour vorhanden ist dann gilt:
    \newline
    \begin{equation}
        x_{ij} = x_{jk} = x_{ki} = 1
    \end{equation}
    Einsetzen (5) in (4)
    \begin{equation}
       \begin{aligned}
           t_i - t_j + 1 &\leq 0 \\
           t_j - t_k + 1 &\leq 0 \\
           t_k - t_i + 1 &\leq 0 \\
       \end{aligned}
    \end{equation}
    \newline
    Addieren alle Formel aus (6) ergibt sich $\Longrightarrow 3 \leq 0$
    \newline
    Die obige Ungleichung ist offensichtlich nicht wahr, daher existiert die Subtour auch nicht.
    Somit ist es klar, dass die Ungleichung (1) grantiert dass es keine Subtour in TSP gibt.
\end{itemize}
\textbf{4. Aufgabe: Praktische Anwendung und Instanzen des TSP}
\newline
\textbf{5. Aufgabe: Anwendung des TSP für Bioinformatikprobleme}
\newline
Welche Bioinformatikprobleme wurden mittels TSP gelöst bzw. untersucht?
Finden und erklären Sie klassische Beispiele!
Was sind die zeitlich neuesten Bioinformatik Paper/Publikationen, die das TSP erwähnen und zur
Lösung nutzen?
Welche neuen Veröffentlichungen gibt es, die versuchen das TSP mit neuen Methoden besser zu
lösen?
\newline
Sequenzierung, Protein vorhersagen...
\newline \\
Ein klassische Bespiele dafür ist Schrotschuss-Sequenzierung im Sequenzierung.
Die Sequenzierung mit dem Schrotschuss-Sequenzierungsverfahren wird in mehrere Phasen eingeteilt:
\begin{itemize}
    \item Fragmentierung der DNA und Sequenzierung der Fragmente (Fragmentierungs-Phase)
    \item Feststellung von Überlappungen zwischen den Fragment-Sequenzen (Overlap-Phase)
    \item Berechnung eines multiplen Alignments der Fragmente (Layout-Phase)
    \item Ermittlung der Konsensus-Sequenz (Konsensus-Phase)
\end{itemize}
Nach Fragmentierungs-Phase müssen die Sequenzen wieder richtig zusammengebaut werden.
Diese Fragmente stammen alle von der Originalsequenz, daher sollte die Originalsequenz zumindest 
alle diese Fragmentsequenzen enthalten. Das heißt, die Originalsequenz ist der Superstring dieser fragmentierten Sequenzen
und ein ``shortest supersequence'' muss davon ausgewählt werden. Das Problem kann durch TSP gelöst werden.
\newline
Shortest Supersequence Problem:
\newline
Gegeben sei eine Buchstabenmenge Menge $\Sigma$, eine endliche Stringmenge $A\subseteq \Sigma^*$,
findet ein String $r$, sodass $\forall x\in A$ ein Teilstring in $r$.
\newline\\
Publikationen und Veröffentlichungen:
\newline
Generation of Genetic Maps Using the Travelling Salesman Problem (TSP) Algorithm
\href{International Journal of Scientific & Engineering Research, Volume 5, Issue 6, June-2014}
\newline
https://www.ijser.org/researchpaper/Generation-of-Genetic-Maps-Using-the-Travelling-Salesman-Problem-TSP-Algorithm.pdf
\newline \\
A (Slightly) Improved Approximation Algorithm for Metric TSP
\newline
Anna R. Karlin, Nathan Klein, Shayan Oveis Gharan
\newline
arXiv.org > cs > arXiv:2007.01409
\newline
Diese neue Veröffentlichung biete eine neue Approximation für TSP mit 3/2 ratio
\end{flushleft}
\end{document}