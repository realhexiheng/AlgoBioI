\documentclass{article}
\usepackage[a4paper,left=10mm,right=10mm,top=15mm,bottom=15mm]{geometry}
\usepackage{amssymb,amsthm,latexsym,amsfonts, amsmath}
\usepackage{extarrows}
\usepackage{german}
\title{Übungen zur Algorithmischen Bioinformatik I\\
Blatt 3}
\author{Xiheng He }
\date{Mai 2021}
\linespread{1.3}
\begin{document}
\maketitle
\begin{flushleft}
\textbf{1. Aufgabe (10 Punkte):}\\
Lösen Sie die folgende inhomogene lineare Rekursionsgleichung mit Hilfe des Satzes zur Lösung homogener 
linearer Rekursionsgleichungen endlicher Ordnung aus der Vorlesung.
\begin{equation*}
    a_n = 2a_{n-1} - a_{n-2} + 1 \text{\qquad und \qquad} a_0 = 0, \quad a_1 = 1.
\end{equation*}
Zeigen Sie mit vollständige Induktion, dass Ihre Lösung richtig ist.
\begin{align}
    a_n &= 2a_{n-1} - a_{n-2} + 1 \\
    a_{n-1} &= 2a_{n-2} - a_{n-3} + 1 \\
    (1) - (2):& \quad a_n - 3a_{n-1} + 3a_{n-2} - a_{n-3} = 0 \quad (\text{Umformung der inhomogenen linearen Rekursionsgleichung})\\
    p(x) &= x^3 - 3x^2 + 3x - 1 = (x-1)^3 \quad (\text{Charakteristische Polynom})
\end{align}
\newline
Aus (4) folgt $x_1 = x_2 = x_3 = 1$, Nach allgeminer Lösung homogener linearer Rekursionsgleichung folgt, dass
\begin{align}
    a_n &= \sum_{i = 1}^{m = 1}\sum_{j = 0}^{2} C_j \cdot n^{\underline{j}} \cdot x^n \\
        &= C_0 \cdot 1 \cdot 1 + C_1 \cdot n \cdot 1 + C_2 \cdot n(n-1) \cdot 1 \quad(C_j := \alpha_{i,j} )  
\end{align}
Aus (6), Mit $a_0 = 0, a_1 = 1$ und $a_2 = 2a_1 - a_0 + 1 = 3$ folgt:
\begin{align}
    a_0 &= C_0 + C_1 \cdot 0 + C2 \cdot 0 = 0 \Longrightarrow C_0 = 0\\
    a_1 & = C_0 + C_1 + C2 \cdot 0 = C_0 + C_1 = 1 \Longrightarrow C_1 = 1\\
    a_2 & = C_0 + 2C_1 + 2C_2 = 3 \Longrightarrow C_2 = \frac{1}{2}\\
    a_n & = n + \frac{n(n-1)}{2} = \frac{n(n+1)}{2}
\end{align}
Induktionsanfang: $n=0 \Longrightarrow a_0 = \displaystyle\frac{0(0+1)}{2} = 0$
\newline
Induktionsvoraussetzung: $a_n = \displaystyle\frac{n(n+1)}{2}$
\newline
Induktionschritt: $n \longrightarrow n+1$
\newline
Induktionsbeweis:
\begin{align*}
    a_{n+1} &= 2a_n - a_{n-1} + 1\\
    &\overset{(10)}{=} 2(\frac{n(n+1)}{2}) - \frac{(n-1)n}{2} + 1\\ 
    &= \frac{2n^2 + 2n - n^2 + n + 2}{2}\\
    &= \frac{n^2 + 3n +2}{2}\\
    &= \frac{(n+1)(n+2)}{2} \qed
\end{align*}
\end{flushleft}
\end{document}