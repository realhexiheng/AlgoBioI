\documentclass{article}
\usepackage[a4paper,left=10mm,right=10mm,top=15mm,bottom=15mm]{geometry}
\usepackage{amssymb,amsthm,latexsym,amsfonts, amsmath, bm}
\usepackage{extarrows}
\usepackage{german}
\title{Übungen zur Algorithmischen Bioinformatik I\\
Blatt 4}
\author{Xiheng He }
\date{Mai 2021}
\linespread{1.5}
\begin{document}
\maketitle
\begin{flushleft}
\textbf{3. Aufgabe (10 Punkte, Notenbonus):}\\
a) Sei SuperComputer ein leistungsfähiger Rechner, der in einer Sekunde 1.000 Elementaroperationen ausführen kann. 
Für ein bestimmtes Problem seien fünf verschiedene Algorithmen verfügbar. Hierbei benötigt der $i$-te Algorithmus 
bei einer Eingabe der Eingabegröße $n$ genau $T_i(n)$ Elementaroperationen, wobei
\begin{align*}
T_1(n) = 500 \cdot n,\ T_2(n) = 50 \cdot n\log_2(n),\ T_3(n) = n^2,\ T_4(n) = \frac{n^3}{100},\ T_5(n) = \frac{3^n}{1000}
\end{align*}
ist. Vervollständigen Sie die folgende Tabelle, in der die Eingabegrößen angegeben sind, für die
der $i$-te Algorithmus auf dem SuperComputer (ziemlich) genau eine Sekunde, eine Minute, eine
Stunde, einen Tag, einen Monat bzw. ein Jahr Rechenzeit benötigt. Lösungsweg mit angeben!
\newline
\bm{$T_1(n)$}: 
\newline
60s: $T_1(n) = 500 \cdot n = 1000 \cdot 60 \Longrightarrow n = 120$; 
3600s: $T_1(n) = 500 \cdot n = 1000 \cdot 3600 \Longrightarrow n = 7200$;
1d = 86400s: $T_1(n) = 500 \cdot n = 1000 \cdot 3600 \cdot 24 \Longrightarrow n = 172800$;
1M = 2592000s: $T_1(n) = 500 \cdot n = 1000 \cdot 3600 \cdot 24 \cdot 30 \Longrightarrow n = 5184000$;
1Y = 31536000s: $T_1(n) = 500 \cdot n = 1000 \cdot 31536000 \Longrightarrow n = 63072000$
\newline
\bm{$T_2(n)$}: 
\newline
1s: $T_2(n) = 50 \cdot n\log_2(n) = 1000 \Longrightarrow 2^{20} = n^n \Longrightarrow n \approx 7$;
60s: $T_2(n) = 50 \cdot n\log_2(n) = 1000 \cdot 60 \Longrightarrow n \approx 163$;
1d = 86400s: $T_2(n) = 50 \cdot n\log_2(n) = 1000 \cdot 86400 \Longrightarrow n \approx 103708$;
\newline
1M = 2592000s: $T_2(n) = 50 \cdot n\log_2(n) = 1000 \cdot 2592000 \Longrightarrow n \approx 2442956$;
\newline
1Y = 31536000s: $T_2(n) = 50 \cdot n\log_2(n) = 1000 \cdot 31536000 \Longrightarrow n^n = \exp (20 \cdot 31536000 )
\Longrightarrow n \approx 25627375$
\newline
\bm{$T_3(n)$}:
\newline
1s: $T_3(n) = n^2 = 1000 \Longrightarrow n = \sqrt{1000} = 31.622 \approx 32$
\newline
60s: $T_3(n) = n^2 = 1000 \cdot 60 \Longrightarrow n = \sqrt{60000} \approx 245$
\newline
3600s: $T_3(n) = n^2 = 1000 \cdot 3600 \Longrightarrow n = \sqrt{1000 \cdot 3600} \approx 1897$
\newline
1d = 86400s: $T_3(n) = n^2 = 1000 \cdot 86400 \Longrightarrow n = \sqrt{86400000} \approx 9295$
\newline
1M = 2592000s: $T_3(n) = n^2 = 1000 \cdot 2592000 \Longrightarrow n = \sqrt{1000 \cdot 2592000} \approx 50912$
\newline
1Y = 31536000s: $T_3(n) = n^2 = 1000 \cdot 31536000 \Longrightarrow n = \sqrt{1000 \cdot 31536000} \approx 117584$ 
\newline
\bm{$T_4(n)$}:
\newline
1s: $T_4(n) = \frac{n^3}{100} = 1000 \Longrightarrow n = \sqrt[3]{10^5} \approx 46$
\newline
60s: $T_4(n) = \frac{n^3}{100} = 1000 \cdot 60 \Longrightarrow n = \sqrt[3]{10^5 \cdot 60} \approx 182$
\newline
3600s: $T_4(n) = \frac{n^3}{100} = 1000 \cdot 3600 \Longrightarrow n = \sqrt[3]{10^5 \cdot 3600} \approx 711$
\newline
1d = 86400s: $T_4(n) = \frac{n^3}{100} = 1000 \cdot 86400 \Longrightarrow n = \sqrt[3]{10^5 \cdot 86400} \approx 2052$
\newline
1M = 2592000s: $T_4(n) = \frac{n^3}{100} = 1000 \cdot 2592000 \Longrightarrow n = \sqrt[3]{10^5 \cdot 2592000} \approx 6376$
\newline
1Y = 31536000s: $T_4(n) = \frac{n^3}{100} = 1000 \cdot 31536000 \Longrightarrow n = \sqrt[3]{10^5 \cdot 31536000} \approx 14665$
\newline
\bm{$T_5(n)$}:
\newline
1s: $T_5(n) = \frac{3^n}{1000} = 1000 \Longrightarrow n = \log_3(10^6) \approx 13$
\newline
3600s: $T_5(n) = \frac{3^n}{1000} = 1000 \cdot 3600 \Longrightarrow n = \log_3(10^6 \cdot 3600) \approx 20$
\newline
1d = 86400s: $T_5(n) = \frac{3^n}{1000} = 1000 \cdot 86400 \Longrightarrow n = \log_3(10^6 \cdot 86400) \approx 23$
\newline
1M = 2592000s: $T_5(n) = \frac{3^n}{1000} = 1000 \cdot 2592000 \Longrightarrow n = \log_3(10^6 \cdot 2592000) \approx 26$
\newline
1Y = 31536000s: $T_5(n) = \frac{3^n}{1000} = 1000 \cdot 31536000 \Longrightarrow n = \log_3(10^6 \cdot 31536000) \approx 28$
\newline\\
\begin{tabular}{||l|c|c|c|c|c|c||}
\hline & 1s & 1m = 60s & 1h = 3.600s & 1d = 86.400s & 2.592.000s & 1Y = 31.536.000s \\
\hline $T_1(n)$ & 2 & 120 & 7200 & 172800 & 5184000 & 63072000 \\
\hline $T_2(n)$ & 7 & 163 & 5763 & 103708 & 2442956 & 25627375 \\
\hline $T_3(n)$ & 32 & 245 & 1897 & 9295 & 50912 & 117584 \\
\hline $T_4(n)$ & 46 & 182 & 711 & 2052 & 6376 & 14665 \\
\hline $T_5(n)$ & 13 & 16 & 20 & 23 & 26 & 28 \\
\hline
\end{tabular}
\newline\\
b) Sie haben sich einen HyperComputer zugelegt, der eine Weiterentwicklung von SuperComputer und 100 mal schneller 
ist, also 100.000 Elementaroperationen pro Sekunde ausführen kann. Um wieviel kann man die Eingabegröße für die fünf 
verschiedenen Algorithmen gegenüber SuperComputer erhöhen, wenn man dieselbe Rechenzeit zur Verfügung hat? 
Diese Veränderung ist (möglichst genau) als Funktion der Eingabegröße (beispielsweise als Faktor oder ähnliches) 
anzugeben, wobei das für $T_2$ nicht genau möglich ist.
\newline
Hinweis: Es ist \textbf{nicht} noch einmal eine Tabelle wie bei Aufgabe 3 anzugeben. Der Lösungsweg muss
nachvollziehbar sein.
\newline
\bm{$T_i(n)$}: SuperComputer \qquad \bm{$T_i(n)'$}: HyperComputer
\begin{align*}
    \bm{T_1(n)} &= 500 \cdot n = 10^3 \land T_1(n)' = (500 \cdot n) \cdot 100 = 10^3 \cdot 100 \Longrightarrow T_1(n)' = 100 \cdot T_1(n) \\
    \bm{T_2(n)} &= 50 \cdot n\log_2(n) = 10^3 \Longrightarrow n^n = \exp(20) \Longrightarrow n \approx 7 \land T_2(n)' = 50 \cdot n'\log_2(n') \\
    &= 10^3 \cdot 100 \Longrightarrow n'^{n'} = \exp(20 \cdot 100) \Longrightarrow n' \approx 251 \Longrightarrow 
    T_2(n)' = 35.857 \cdot T_2(n) \approx 36 \cdot T_2(n) \\
    \bm{T_3(n)} &= n^2 = 1000 \Longrightarrow n = 10 \sqrt{10} \land T_3(n)' = n'^2 = 1000 \cdot 100 \Longrightarrow n' = 10^2 \sqrt{10} 
    \Longrightarrow T_3(n)' = 10 \cdot T_2(n) \\
    \bm{T_4(n)} &= \frac{n^3}{100} = 1000 \Longrightarrow n^3 = 10^5 \land T_4(n)' = \frac{n'^3}{100} \Longrightarrow n'^3 = 10^7 
    \Longrightarrow \frac{n'}{n} = \sqrt[3]{100} = 4.6416 \Longrightarrow T_4(n)' \\
    & \approx 4.64 \cdot T_4(n) \\
    \bm{T_5(n)} &= \frac{3^n}{1000} = 1000 \Longrightarrow n = \log_3 10^6 \land T_5(n)' = \frac{3^{n'}}{1000} = 1000 \cdot 100 \Longrightarrow n' = \log_3 10^8
    \Longrightarrow \frac{n'}{n} = \frac{\log_3 10^8}{\log_3 10^6} = \frac{4}{3} \\
    & \Longrightarrow T_5(n)' = \frac{4}{3} \cdot T_5(n) \approx 1.33 \cdot T_5(n)
\end{align*}
\end{flushleft}
\end{document}