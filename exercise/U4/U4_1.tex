\documentclass{article}
\usepackage[lined,ruled]{algorithm2e}
\usepackage[a4paper,left=10mm,right=10mm,top=15mm,bottom=15mm]{geometry}
\usepackage{amssymb,amsthm,latexsym,amsfonts, amsmath}
\usepackage{german}
\title{Übungen zur Algorithmischen Bioinformatik I\\
Blatt 4}
\author{Xiheng He }
\date{April 2021}
\linespread{1.5}
\SetKwFor{For}{for (}{)\text{ do}}{}
\SetKw{int}{int}
\begin{document}
\maketitle
\begin{flushleft}
\textbf{1. Aufgabe (10 Punkte):}
\renewcommand\footnoterule{}
\begin{algorithm}
    \NoCaptionOfAlgo
    \caption{SumExist(int[] S, int x, int n)}
    \Begin{
    \SetAlgoVlined
    \int $l = 0$\;
    \int $r = n - 1$\;
    \tcc{If $x = sum$ loop ends, $l$ must be smaller than $r$} 
    \While{$(l \leq r)$}{
        \If{$(S[l] + S[r] == x)$}{
            \Return{\textbf{true}\;}
            }
        \tcc{if $x < sum$, reduce the sum until equal to $x$} 
        \ElseIf{$(S[l] + S[r] > x)$}
        {$r--$\;}
        \tcc{if $x > sum$, increase the sum until equal to $x$} 
        \Else{$l++$\;}
        \Return{\textbf{false}\;}
    }
    }
\end{algorithm}\\
\textbf{Analyse:} \\
Eingabegröße: $n = |S|$ \\
Laufzeit: Vergleich benötigt genau ein mal Pro Rekursion und nach Vergleich wird entweder $l$ um 1 erhöht oder $r$ um 1 reduziert. 
Egal wie l oder r sich verändert, die Kosten der arithmetischen Operationen beträgt jeweils $l$ und $n-1-r$.
Damit ist die gesamte Kosten $l+n-1-r$ wobei $l-r \leq 0$ und While-Schleife kann maximal $n$ mal durchlaufen. 
Somit ist die Laufzeit $O(n)$.
\newline
Korrektheit: Der Algorithmus wird am Anfang mit dem Summe der ersten und letzten beiden Elementen und $x$ vergleichen.
Dazu gibt es drei Fälle: Wenn $x = S[0] + S[n-1]$ wird $true$ zurückgegeben; Wenn $x > S[0] + S[n-1]$ bedeutet es dass 
die Summe erhöht werden muss, dann wird ein Element großer als das kleineste Element $S[0]$ benötigt. Anderfalls wird ein kleinere Element $<S[n-1]$ benötigt.
Ist ein großer Element oder große Summe nötig, muss das nach rechts gesucht werden, sonst nach links. Damit wird $l$ steigen und $r$ verringert bis $l=r$.
Jedes Mal wenn sich der Pointer $l$ oder $r$ bewegt checken wir ob $S[l] + S[r] == x$.
Wenn solche zwei Elemente am Ende noch nicht gefunden wurden, dann gibt überhaupt nicht. 
\end{flushleft}
\end{document}