\documentclass{article}
\usepackage[a4paper,left=5mm,right=5mm,top=15mm,bottom=15mm]{geometry}
\usepackage{amssymb,amsthm,latexsym,amsfonts, amsmath, bm}
\usepackage{extarrows}
\usepackage{enumerate}
\usepackage{tikz}
\usepackage{tikz-qtree}
\usepackage{graphicx}
\usepackage{caption} 
\usepackage{subcaption}
\usepackage{german}
\title{Übungen zur Algorithmischen Bioinformatik I\\
Blatt 10}
\author{Xiheng He}
\date{Juni 2021}
\linespread{2.0}
\newtheorem{theorem}{Theorem}[section]
\begin{document}
\maketitle
\begin{flushleft}
\textbf{3. Aufgabe: String-Matching (10 Punkte)}
\begin{enumerate}[(a)]
    \item Zeigen Sie die Arbeitsschritte der Algorithmen an folgendem Beispiel:
    \begin{equation*}
        P = {\rm ACACBDA}, T = {\rm ACACEDAACACBDAA}
    \end{equation*}
    \begin{itemize}
        \item des naiven Algorithmus,
        \newline 
        Sehen Abbildung 1.
        \begin{figure}
            \centering
            \begin{subfigure}{0.3\textwidth}
                \centering
                \includegraphics[width=\textwidth]{naive_1.png}
                \caption{Schritt 1}
            \end{subfigure}
            \begin{subfigure}{0.3\textwidth}
                \centering
                \includegraphics[width=\textwidth]{naive_2.png}
                \caption{Schritt 2}
            \end{subfigure}
            \begin{subfigure}{0.3\textwidth}
                \centering
                \includegraphics[width=\textwidth]{naive_3.png}
                \caption{Schritt 3}
            \end{subfigure}
            \begin{subfigure}{0.3\textwidth}
                \centering
                \includegraphics[width=\textwidth]{naive_4.png}
                \caption{Schritt 4}
            \end{subfigure}
            \begin{subfigure}{0.3\textwidth}
                \centering
                \includegraphics[width=\textwidth]{naive_5.png}
                \caption{Schritt 5}
            \end{subfigure}
            \begin{subfigure}{0.3\textwidth}
                \centering
                \includegraphics[width=\textwidth]{naive_6.png}
                \caption{Schritt 6}
            \end{subfigure}
            \begin{subfigure}{0.3\textwidth}
                \centering
                \includegraphics[width=\textwidth]{naive_7.png}
                \caption{Schritt 7}
            \end{subfigure}
            \begin{subfigure}{0.3\textwidth}
                \centering
                \includegraphics[width=\textwidth]{naive_8.png}
                \caption{Schritt 8: Match gefunden}
            \end{subfigure}
        \caption{Naive Algorithmus}
        \end{figure}
        \item des Knuth-Morris-Pratt-Algorithms (KMP)
        \newline
        Sehen Abbildung 2.
        \newline \\
        \begin{tabular}{||l|c|c|c|c|c|c|c||}
            \hline j & 0 & 1 & 2 & 3 & 4 & 5 & 6 \\
            \hline s[j] & A & C & A & C & B & D & A \\
            \hline border[j] & 0 & 0 & 1 & 2 & 0 & 0 & 1 \\
            \hline
        \end{tabular}
        \newline
        \begin{figure}
            \centering
            \begin{subfigure}{0.3\textwidth}
                \centering
                \includegraphics[width=\textwidth]{kmp_1.png}
                \caption{Schritt 1: j = 4, border[4] = 2, j - border[4] = 2}
            \end{subfigure}
            \begin{subfigure}{0.3\textwidth}
                \centering
                \includegraphics[width=\textwidth]{kmp_2.png}
                \caption{Schritt 2: j = 2, border[2] = 0, 2 - border[2] = 2}
            \end{subfigure}
            \begin{subfigure}{0.3\textwidth}
                \centering
                \includegraphics[width=\textwidth]{kmp_3.png}
                \caption{Schritt 3: j = 0, border[0] = -1, j - border[0] = 1}
            \end{subfigure}
            \begin{subfigure}{0.3\textwidth}
                \centering
                \includegraphics[width=\textwidth]{kmp_4.png}
                \caption{Schritt 4: j = 0, border[0] = -1, j - border[0] = 1}
            \end{subfigure}
            \begin{subfigure}{0.3\textwidth}
                \centering
                \includegraphics[width=\textwidth]{kmp_5.png}
                \caption{Schritt 5: j = 1, border[1] = 0, j - border[1] = 1}
            \end{subfigure}
            \begin{subfigure}{0.3\textwidth}
                \centering
                \includegraphics[width=\textwidth]{kmp_6.png}
                \caption{Schritt 6: Match gefunden}
            \end{subfigure}
        \caption{KMP Algorithmus}
        \end{figure}
        \item des Boyer-Moore-Algorithmus (BM) nur mit next-Tabelle (BM\_next)
        \newline
        Sehen Abbildung 3.
        \newline \\
        \begin{tabular}{||l|c|c|c|c|c|c|c|c||}
            \hline j & 0 & 1 & 2 & 3 & 4 & 5 & 6 & 7 \\
            \hline S[j] & 6 & 6 & 6 & 6 & 6 & 4 & 1 & 7 \\
            \hline
        \end{tabular}
        \newline
        \begin{figure}
            \centering
            \begin{subfigure}{0.3\textwidth}
                \centering
                \includegraphics[width=\textwidth]{bm_next_1.png}
                \caption{Schritt 1: j = 4, S[j] = 6, shift = 6}
            \end{subfigure}
            \begin{subfigure}{0.3\textwidth}
                \centering
                \includegraphics[width=\textwidth]{bm_next_2.png}
                \caption{Schritt 2: j = 6, S[j] = 1, shift = 1}
            \end{subfigure}
            \begin{subfigure}{0.3\textwidth}
                \centering
                \includegraphics[width=\textwidth]{bm_next_3.png}
                \caption{Schritt 3: Match gefunden}
            \end{subfigure}
        \caption{BM\_next Algorithmus} 
        \end{figure}
        \item und Boyer-Moore nur mit bad-character-rule (BM\_bad-char)
        \newline
        Sehen Abbildung 4.
        \newline \\
        \begin{tabular}{||l|c|c|c|c||}
            \hline a & A & C & B & D \\
            \hline ebc[a] & 6 & 3 & 4 & 5 \\
            \hline
        \end{tabular}
        \newline
    \end{itemize}
    Erstellen Sie dazu auch die zugehörigen $next$ (improved) und $skip$ Tabellen. Machen Sie deutlich,
    welche Verschiebungen und Vergleiche durchgeführt werden (z.B durch Zeichnungen!).
    \newpage
    \item Entscheidend für die ``besseren'' Shifts ist die Nutzung von Informationen über das Pattern, die
    durch die Zeichenvergleiche gewonnen wurden:
    \newline
    Wenn ein $q$-Prefix ($q] P$) des Patterns $P$ und der Stelle $s + 1$ im Text $T$ matcht, also
    \begin{equation*}
        q] P := P[ 1,\dots,q]  = T[ s + 1,\dots,s + q] 
    \end{equation*}
    dann soll diese Information für den Shift ausgenutzt werden, so daß $P[1,\dots k] = T[s' + 1,\dots,s' + k]$ und
    $s' + k = s + q$ (also ein $k$-Prefix $k]P$ von $P$ matcht ein $k$-Suffix ($k[T$) von $T$). Da aber dieses $k$-Suffix
    von $T$ ein $k$-Suffix des $q$-Prefixes von $P$ ist, kann des vorberechnet werden durch Vergleich von $P$ gegen sich
    selbst.
    \newline
    Zeigen Sie, dass der Shift $\pi$ im KMP Algorithmus für jedes $q$ genau die (Länge der) maximalen Prefixe von $q]P$
    berechnet, die Suffixe von $q]P$ matchten:
    \begin{equation*}
        \pi[q] = max\{k < q | Q = P]q \land k]Q = k[Q\}
    \end{equation*} 
    \newline
    In KMP Algorithmus ist Shift $\pi$ wie folgende definiert:
    \newline
    $\pi[q] := j - border[j]$ wobei $j$ ist die Position wo Mismatch vorkommt und $border[j]$ der Länge der Prefix und auch Suffix.
    \newline \\
    $q] P := P[ 1,\dots,q]  = T[ s + 1,\dots,s + q] \Longrightarrow$ Mismatch kommt an q + 1 vor und $k$ sei Länge der Prefix.
    $\Longrightarrow \pi[q] := q + 1 - k$ 
    \newline \\
    Jeder Shift berechnet genau die maximalen Prefixe da Prefixe erweitert werden können somit werden die Prefixe mit maximalen Länge
    nicht übersehen. D.h. $\pi[q] = max\{k < q | Q = P]q\}$
    Gibt's ein von Shift berechnete Prefix, dann gibt näturlich auch ein Suffix in $P$, so dass zwei Teilfolgen übereinstimmen. D.h. 
    $k]Q = k[Q$ Sonst wurde kein Prefix ausgegeben.
    \newline \\
    offensichtlich, dass $k < q$, sonst wurde statt Prefix ein Suffix mit eine Teilfolge von $T$ übereinstimen somit ein Match schon vorgekommen ist.  
\end{enumerate}
\end{flushleft}
\end{document}