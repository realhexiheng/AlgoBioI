\documentclass{article}
\usepackage[a4paper,left=10mm,right=10mm,top=15mm,bottom=15mm]{geometry}
\usepackage{amssymb,amsthm,latexsym,amsfonts, amsmath, bm}
\usepackage{extarrows}
\usepackage{german}
\title{Übungen zur Algorithmischen Bioinformatik I\\
Blatt 5}
\author{Xiheng He }
\date{Mai 2021}
\linespread{1.5}
\begin{document}
\maketitle
\begin{flushleft}
\textbf{3. Aufgabe: ’Collatz’-Zahlenfolge (10 Punkte, Bonus)}\\
Eine Folge $a_0,a_1,\dots a_n$ natürliche Zahlen ist durch einen Startwert $a_0 > 1$ und die 
folgende Vorschrift (3-5\_Collatz) bestimmt:
\begin{equation}
a_{i+1} = \left\{
\begin{aligned}
    &a_i \div 3 \qquad\text{falls $a_i$ (ohne Rest) durch 3 teilbar} \\
    &a_i + 5 \qquad\text{sonst}
\end{aligned}
\right.
\end{equation}
Für welche Startwerte $a_0$ gibt es ein $n$ mit $a_n$ = 1?
$$
\begin{aligned}
    &a_0 = 2: 2, 7, 12, 4, 9, 3, 1 \\
    &a_0 = 3: 3, 1 \\
    &a_0 = 4: 4, 9, 3, 1 \\
    &a_0 = 5: 5, 10, 15, 5, 10, 15, 5, 10, 15,\dots \Longrightarrow 5-10-15-5 \\
    &a_0 = 6: 6, 2, 7 ,12, 4, 9, 3, 1 \\
    &a_0 = 7: 7, 12, 4, 9 , 3, 1 \\
    &a_0 = 8: 8, 13, 18, 6, 2, 7, 12, 4, 9, 3, 1 \\
    &a_0 = 9: 9, 3, 1 \\
    &a_0 = 10: 10, 15, 5, 10,\dots \Longrightarrow 5-10-15-5 \\
    &a_0 = 11: 11, 16, 21, 7, 12, 4, 9, 3, 1 \\
    &a_0 = 12: 12, 4, 9, 3, 1 \\
    &a_0 = 13: 13, 18, 6, 2, 7, 12, 4, 9, 3, 1 \\
    &a_0 = 14: 14, 19, 24, 8, 13, 18, 6, 2, 7, 12, 4, 9, 3, 1 \\
    &a_0 = 15: 15, 5, 10, 15,\dots \Longrightarrow 5-10-15-5 \\
    &a_0 = 16: 16, 21, 7, 12, 4, 9, 3, 1 \\
    &a_0 = 17: 17, 22, 27, 9, 3, 1 \\
    &a_0 = 18: 18, 6, 2, 7, 12, 4, 9, 3, 1 \\
    &a_0 = 19: 19, 24, 8, 13, 18, 6, 2, 7, 12, 4, 9, 3, 1 \\
    &a_0 = 20: 20, 25, 30, 10, 5, 10, 15, 5, 10,\dots \Longrightarrow 5-10-15-5
\end{aligned}
$$
Vermutung 1: Für alle näturliche Zahlen $a_0 > 1$, die durch 5 teilbar sind, werden in den Zyklus 5, 10, 15, 5 
übergehen und können sich nicht beenden.
\newline\\
Vermutung 2: Für alle natürliche Zahlen $a_0 > 1$, die nicht durch 5 teilbar sind, werden in den Zyklus 9, 3, 1 
übergehen und können sich an 1 beenden.
\newline \\
Beweisen Sie Ihre Vermutung, zum Beispiel durch vollständige Induktion!
\newline
Vermutung 1:
\begin{itemize}
    \item Induktionsanfang: $a_0 = 5: 5, 10, 15, 5, 10, 15, 5, 10, 15,\dots \Longrightarrow \textbf{Zyklus}: 5-10-15-5$ \\
    \item Induktionsvoraussetzung: Für alle näturliche Zahlen $a_0 > 1$, die durch 5 teilbar sind, werden in den Zyklus 5, 10, 15, 5 
    übergehen und sich nicht beenden.
    \item Induktionschritt: $n \longrightarrow n+1$ 
    \item Induktionsbeweis: $a_n, a_{n + 1}$ sei näturliche Zahlen und sind durch 5 teilbar. \\
    $$ a_{n + 1} = \left\{
    \begin{aligned}
        5 + 5 = 10 \overset{(1)}{\Longrightarrow} 15, 5, 10, 15, 5 \Longrightarrow \textbf{Zyklus}: 5-10-15-5 
        &\qquad\text{falls } a_n = 5 \\
        10 + 5 = 15 \overset{(1)}{\Longrightarrow} 5, 10, 15, 5 \Longrightarrow \textbf{Zyklus}: 5-10-15-5 
        &\qquad\text{falls } a_n = 10 \\
        15 + 5 = 20  \overset{(1)}{\Longrightarrow} 25, 30, 10, 5, 10, 15, 5 \Longrightarrow \textbf{Zyklus}: 5-10-15-5 
        &\qquad\text{falls } a_n = 15 \qed
    \end{aligned}
    \right.
    $$ \\
\end{itemize}
Vermutung 2:
$a_n, a_{n + 1}$ sei näturliche Zahlen und sind nicht durch 5 teilbar. Setze $f(a_n) := a_{n + 1}$ d.h. $f(a_n)$ sei 
’Collatz’ Funktion. 

\begin{itemize}
    \item Induktionsanfang $a_0 = 2: 2, 7, 12, 4, 9, 3, 1 \land a_1 = 7, 12, 4, 9, 3, 1\Longrightarrow \textbf{Zyklus}: 9-3-1 \land \textbf{endet an } 1$
    \item Induktionsvoraussetzung: Für alle natürliche Zahlen $a_0 > 1$, die nicht durch 5 teilbar sind, können sich an 1 beenden.
    \item Induktionschritt: $n \longrightarrow n+1$ 
    \item Induktionsbeweis: 
    $$ a_{n + 1} = f(a_n)\left\{
    \begin{aligned}
        &f(9) \overset{(1)}{\Longrightarrow} f(3) \overset{(1)}{\Longrightarrow} 1 \Longrightarrow \textbf{Zyklus}: 9-3-1 \land \textbf{endet an } 1
        \qquad\text{falls } a_n = 9 \\
        &f(3) \overset{(1)}{\Longrightarrow} 1 \underset{\text{abgeleitet werden}}{\overset{\text{3 kann nur aus 9}}{\Longrightarrow}} \textbf{Zyklus}: 9-3-1
        \land \textbf{endet an } 1 
        \qquad\text{falls } a_n = 3 \qed
    \end{aligned}
    \right.
    $$ \\
     
\end{itemize}
Was ist der Unterschied zur Collatz-Folge, die Sie aus der Vorlesung kennen?
Hinweis: Überlegen Sie sich zunächst, in welchen Fällen die Reihe konvergiert bzw. nicht konvergiert.
\newline
\begin{itemize}
    \item Die Vorschrift ist unterschiedlich. In dieser Alternative eine natürliche Zahl wird entweder durch 3 geteilt 
    falls sie ohne Rest teilbar ist oder 5 addiert 5. In originaler Collatz-Folge wird eine nicht gerade Zahl mit 3 
    multipliziert und 1 addiert und eine gerade Zahl durch 2 geteilt.
    \item In dieser Alternative ist es gezeigt, dass für alle natürliche Zahlen die durch 5 teilbar sind, sich nicht beenden 
    können. In originaler Collatz-Folge wurden noch keine endlose Zahlenfolge gefunden.
    \item In dieser Alternative gibt ein Zyklus 9, 3, 1 für endliche Zahlenfolge und ein Zyklus 5, 10, 15, 5 für endlose 
    Zahlenfolge. In originaler Collatz-Folge endet eine Zahlenfolge in den Zyklus 4, 2, 1.
\end{itemize}
\end{flushleft}
\end{document}