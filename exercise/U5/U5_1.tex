\documentclass{article}
\usepackage[a4paper,left=10mm,right=10mm,top=15mm,bottom=15mm]{geometry}
\usepackage{amssymb,amsthm,latexsym,amsfonts, amsmath, bm}
\usepackage{extarrows}
\usepackage{german}
\title{Übungen zur Algorithmischen Bioinformatik I\\
Blatt 5}
\author{Xiheng He }
\date{Mai 2021}
\linespread{1.5}
\begin{document}
\maketitle
\begin{flushleft}
\textbf{1. Aufgabe: Geschlossene Probleme (closed problems) (10 Punkte)}\\
(a) In der Vorlesung haben Sie verschiedene Algorithmen zum Sortieren von Listen von Elementen
bei gegebener Vergleichsoperation ($<, \leq$) auf den Elementen kennengelernt ($=$ Problem SORT).
\newline
Ist das SORT-Problem closed?
\newline
SORT-Problem ist closed.
\newline
Erklären Sie was closed bedeutet!
\newline
Wenn die entsprechende Algorithmen für solche Probleme einerseits eine inhärente Laufzeit-Komplexität (z.B. Mergesorts is $O(n\log n)$) 
und andererseits eine untere Schranke $\Omega(n\log n)$ haben, sind diese Probleme ”closed problem”. 
\newline
Beweisen oder widerlegen sie closed(SORT)!
\newline
Gegeben sei genau $n$ unterschiedliche Zahlen. In best-case wurden sie alle schon sortiert damit 
wird dies nicht betrachtet. In worst-case sind alle nicht sortiert dann nehmen wir an, dass die Algorithmen 
den richtige Ergebnis liefern nur dann wenn sie $x$ Schritte berechnet haben nämlich $2^x$ Vergleichsoperationen durchführen müssen,
weil für jeden Schritt zwei Elemente verglichen werden deshalb es nur zwei mögliche Fälle für jeden Vergleich gibt. 
\newline  
Ferner sind theoritsch auch $n!$ Permutationen für solche Sotierprobleme vorhanden und Algorithmen benötigen höchstens 
$2^x$ Vergleichsoperationen um die einzige richtige Permutation herauszufinden. Deshalb folgt:
\begin{align*}
    & 2^x  \geq n! \Longrightarrow x \geq \log(n!) \\
    & \because n! = n(n-1)(n-2) \dots \frac{n}{2} \dots 1 \geq (\frac{n}{2})^{\frac{n}{2}} \\
    & \therefore \log(n!) \geq \log((\frac{n}{2})^{\frac{n}{2}}) \\
    &= \frac{n}{2} \cdot \frac{\log n}{\log 2} \\
    &= \Theta(n\log n) \\
    & \Longrightarrow x = \Omega(n\log n)
\end{align*}
Daraus können wir Schlussfolgerungen ziehen, dass die untere Schranke für Sortieralgorithmen $\Omega(n\log n)$ ist.
Zusammengefasst ist eine obere Schranke $O(n\log n)$ für SORT-Problem schon bekannt und eine untere Schranke $\Omega(n\log n)$
schon beweist dann wird solche Problem als ``closed-problem” genannt.
\newpage
(b) Auch das Traveling Salesperson Problem (TSP) kennen Sie z.B. Aus der Vorlesung.
\newline
Ist closed(TSP)?
\newline
TSP ist nicht closed.
\newline
Beweisen oder widerlegen sie closed(TSP)!
\newline
TSP ist als ``NP-hard” Problem bekannt, d.h. kein Algorithmus existiert, der eine optimaleste Reiseroute in polynomieller 
Laufzeit bestimmt. Es gibt zwar konkrete Algorithmen, z.B berechnen alle Permutationen mit $O(n!)$ Komplexität und Held–Karp algorithm 
mit $O(n^2 2^n)$ Komplexität aber wir können kein $k$ festlegen, ob ein Algorithmus mit $\Omega(n^k)$ vorhanden ist. Daher 
existiert lower-bound nicht. Weiterhin können wir ein theoritsches lower-bound $\Omega(n^2)$ herleiten, da der TSP Algorithmus mit $n(n-1)/2$ 
Entfernungen für $n$ Städten umgehen muss jedoch ist dies sinnlos da wir noch keinen Algorithmus in polynomieller Laufzeit gefunden haben.
\newline
\end{flushleft}
\end{document}