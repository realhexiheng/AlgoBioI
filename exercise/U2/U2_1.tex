\documentclass{article}
\usepackage[lined,ruled]{algorithm2e}
\usepackage[a4paper,left=10mm,right=10mm,top=15mm,bottom=15mm]{geometry}
\usepackage{amssymb,amsthm,latexsym,amsfonts, amsmath}
\usepackage{german}
\title{Übungen zur Algorithmischen Bioinformatik I\\
Blatt 2}
\author{Xiheng He }
\date{April 2021}
\linespread{1.5}
\SetKwFor{For}{for (}{)\text{ do}}{}
\begin{document}
\maketitle
\begin{flushleft}
\textbf{1. Aufgabe (10 Punkte):}
\renewcommand\footnoterule{}
\begin{algorithm}
    \NoCaptionOfAlgo
    \caption{FIND(int[] a, int x, int start, int end)}
    \Begin{
    \SetAlgoVlined
    int $m := \lfloor\frac{start + end}{2}\rfloor$;\par
    \If{$(a[m] > x)$} 
    { 
        \Return{\rm FIND$(a, x,start, m)$}\tcc*{If $a[m]$ is greater than $x$ then return the left half}
    }
    \eIf{$(a[m] < x)$}
    {
        \Return{\rm FIND$(a, x, m + 1, end)$}\tcc*{If $a[m]$ is smaller than $x$ then return the right half}
    }
    {\Return{$m$}\tcc*{$a[m] == x$ and $m$ is the position}
    }
    }
\end{algorithm}\\
\textbf{Analyse:} \\
Eingabegröße: ||A|| = $n$. \\
Charakteristischen Operationen: > und <. \\
Laufzeit: Vergleich braucht einmal Pro Rekursion. Somit ist die Laufzeit: 
    $$ A_{\rm DC}(n)=\left\{
    \begin{aligned}
    = & 1 &\text{falls} \quad n = 1 \\
    = & A_{\rm DC}(\lfloor n/2 \rfloor) + 1 &\text{falls} \quad n > 1\\
    \end{aligned}
    \right.
    $$
Angenommen, dass $n = 2^k$:
$$ A_{\rm DC}(n) = A(n/2) + 1 = A(n/4) + 2 = A(n/8) + 3 = A(2^k/2^k) + k = A(1) + log(n) = 1 + log(n) $$
Deshalb: $A_{\rm DC}(n) \in \Theta(log(n))$. \\
Korrektheit: Sei A sortiert, es gibt immer nur drei Fälle in jeder rekursion: $x = a[m]$, $x < a[m]$ und $x > a[m]$. 
Wenn $x$ nicht $a[m]$ gleich ist, wird $x$ in nächster rekursion weiter gesucht. Wenn $x$ kleiner als $a[m]$ ist und
nicht in der linken Hälfte gefunden werden kann, ist die linke Hälfte offensichtlich nicht sortiert. 
Daher wiederspricht es und damit ist der Algorithmus korrekt. Analog kann man die rechte Hälfte auch beweisen wenn $x$ großer als $a[m]$.
\end{flushleft}
\end{document}