\documentclass{article}
\usepackage[a4paper,left=15mm,right=15mm,top=15mm,bottom=15mm]{geometry}
\usepackage{amssymb,amsthm,latexsym,amsfonts, amsmath, bm}
\usepackage[lined,ruled]{algorithm2e}
\usepackage{extarrows}
\usepackage{enumerate}
\usepackage{tikz}
\usepackage{german}
\usepackage{underoverlap}
\usepackage{xcolor}
\newtheorem{lemma}{Lemma}
\newtheorem{theorem}{Theorem}
\title{Übungen zur Algorithmischen Bioinformatik I\\
Blatt 11}
\author{Xiheng He}
\date{Juli 2021}
\linespread{1.5}
\newUOLdecorator\UOLfbox{{%
    \fboxsep=1.2pt\fboxrule=.8pt%
    \kern-\fboxrule \kern-\fboxsep%
    {\color{red}
    \fbox{\color{red}$#1$}}%
    \kern-\fboxsep  \kern-\fboxrule%
}}
\newUOLdecorator\UOLfboxx{{%
    \fboxsep=1.2pt\fboxrule=.5pt%
    \kern-\fboxrule \kern-\fboxsep%
    {\color{blue}
    \fbox{\color{blue}$#1$}}%
    \kern-\fboxsep  \kern-\fboxrule%
}}
\begin{document}
\maketitle
\begin{flushleft}
\textbf{1. Aufgabe: Z-Boxes and Z-Algorithm (10 Punkte)}
\newline
Fundamental preprocessing of a string computes so called \textbf{Z-boxes} and yields probably the simplest
linear-time exact matching algorithm.
\newline \\
\textbf{Definition (Z-Box)}: Given a string $S$ and a position $i > 1$ in $S$. $Z_i(S)$ is the \textbf{length} of the longest
substring in $S$ that starts at $i$ and matches a prefix of $S$.
\begin{equation*}
    Z_i(S) = max_k \{(i[S) [1 : k] = k]S \} = max_k \{S[i : i + k] = S[1 : k]\}
\end{equation*}
Given the following binary string $B = 010100101011$
\begin{enumerate}[(a)]
    \item For all $i$, compute the $Z_i(B)$ values.
    \newline \\
    \begin{tabular}{|c||c|c|c|c|c|c|c|c|c|c|c|c|}
        \hline i & 1 & 2 & 3 & 4 & 5 & 6 & 7 & 8 & 9 & 10 & 11 \\
        \hline z-box & - & 010 & - & 0 & 01010 & - & 0101 & - & 01 & - & - \\
        \hline z-value & 0 & 3 & 0 & 1 & 5 & 0 & 4 & 0 & 2 & 0 & 0 \\
        \hline
    \end{tabular}
    \item Draw all Z-boxes (i.e.$Z_i(B)$) > 0 for string $B$.
    \newline
    \emph{Hinweis}: Boxen mit blauer Farbe sind Z-boxen.
    \begin{itemize}
        \item $Z_2 = 3: \qquad \UOLfbox{0\;1\;}[0]\UOLfboxx{\;1\;0} \;0\;1\;0\;1\;0\;1\;1$
        \item $Z_4 = 1: \qquad \UOLfbox{0}\;1\; 0\; 1\; \UOLfboxx{0}\; 0\; 1\; 0\; 1\; 0\; 1\; 1$
        \item $Z_5 = 5: \qquad \UOLfbox{0\;1\; 0\; 1\;0}\;\UOLfboxx{ 0\; 1\; 0\; 1\; 0}\; 1\; 1$
        \item $Z_7 = 4: \qquad \UOLfbox{0\;1\; 0\; 1}\;0\; 0\; 1\;\UOLfboxx{0\; 1\; 0\; 1}\; 1$
        \item $Z_9 = 2: \qquad \UOLfbox{0\;1}\;0\;1\;0\; 0\; 1\;0\; 1\; \UOLfboxx{0\;1}\; 1$
    \end{itemize}
    \item What is $Z_1(S)$?
    $$ Z_1(S) = \left\{
    \begin{aligned}
        & k + 1 \qquad \text{falls} \; S[0 : k] = S[1 : 1 + k] \\
        & 0 \qquad \text{sonst}
    \end{aligned}  
    \right.
    $$  
    \item What is the complexity of a naive algorithm to compute all Z-Boxes of a string $S$ of length
    $n$? Explain!
    \newline
    Für einen naiven Algorithmus liegt die Komplexität der Laufzeit in $O(m^2)$ wobei $m$ die Länge der Strings $S$ ist.
    \begin{algorithm}
        \NoCaptionOfAlgo
        \caption{Z-Box\_naive (int[] S, int m)}
        \Begin{
        \SetAlgoVlined
        \For{$(i = 1; i < m; i\textbf{\rm\scriptsize++})$}{ 
            $Z_i := 0, j : = 0$ \tcc*{start from 0} \par
            \While{$((i + j < m) \; \& \&  \; (S[j] = S[i + j]))$}
            {
                $Z_i := Z_i + j$; \par
                $j := j + 1$;
            }
        }
    } 
    \end{algorithm}
    Beide While-Schleife und For-Schleife können maximal $m - 1$ mal durchlaufen. Somit liegt die Komplexität der Laufzeit in $O(m^2)$ 
    für einen naiven Algorithmus (siehe Pseudocode).
    \newpage
    \item Let $Z_i(S) > 1$ what is the end $k$ of the Z-Box at position $i$, i.e. $Z[i,\dots,k]$.
    \newline Leider kann man hier nicht wirklich nachvollziehen, wofür $k$ steht. In \textbf{Definition (Z-Box)} ist $Z_i(S) = S[i : i + k] = S[1 : k]$,
    aber hier ist $Z_i(S) = Z[i,\dots,k]$. Exakter Wert kann nicht ausgegeben werden aber wenn $Z_i(S) > 1$ gilt:
    \begin{equation*}
        Z_i(S) > 1 \Longrightarrow |Z[i,\dots,k]| > 1 \Longrightarrow k > i \land k < |S|
    \end{equation*} 
\end{enumerate}
Führen Sie per Hand den Z-Box Algorithmus für $B = 010100101011$ durch. Geben Sie in einer Tabelle
für jedes $k$ die Werte von $k', q, |\beta|, l_k, r_k$ und $Z_k$ in dieser Reihenfolge an. 
Falls einige dieser Werte für einen Schritt irrelevant sind, kennzeichnen Sie dies.
\newline
\textbf{Achtung: $k' \not = k - l + 1, k' = k - l$}
\newline
\textbf{irrelevant Variablen: -}
\newline \\
    \begin{tabular}{|c||c|c|c|c|c|c|c|c|c|c|c|c|}
        \hline $k$ & 1 & 2 & 3 & 4 & 5 & 6 & 7 & 8 & 9 & 10 & 11 \\
        \hline $Z_k$ & 0 & 3 & 0 & 1 & 5 & 0 & 4 & 0 & 2 & 0 & 0 \\
        \hline $l_k$ & 0 & 2 & 2 & 4 & 5 & 5 & 7 & 7 & 9 & 9 & 9 \\
        \hline $r_k$ & 0 & 4 & 4 & 4 & 9 & 9 & 10 & 10 & 10 & 10 & 10 \\
        \hline $k'$ & - & - & 1 & 0 & - & 1 & 2 & 1 & 2 & 1 & 2 \\
        \hline $|\beta|$ & - & - & 2 & 1 & - & 4 & 4 & 3 & 2 & 1 & 0 \\
        \hline $q$ & - & - & - & - & - & - & 11 & - & 11 & - & 11 \\
        \hline  
    \end{tabular}
\end{flushleft}
\end{document}