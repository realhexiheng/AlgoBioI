\documentclass{article}
\usepackage[a4paper,left=10mm,right=10mm,top=15mm,bottom=15mm]{geometry}%\usepackage{a4wide}
\usepackage{amssymb,amsthm,latexsym,amsfonts, amsmath, bm}
\usepackage{extarrows}
\usepackage{enumerate}
\usepackage{german}
\title{Übungen zur Algorithmischen Bioinformatik I\\
Blatt 6}
\author{Xiheng He }
\date{Mai 2021}
\linespread{2.0}
\begin{document}
\maketitle
\begin{flushleft}
\textbf{2.Aufgabe: (Amortisierte) Algorithmenanalyse (10 Punkte)}
\newline
Betrachten Sie folgendes einfache Problem (BinaryCounter):
\newline
Zählen Sie von 0 bis $n$ mittels einer binären String-Darstellung des Zählers, z.B. i = 0 . . . 0010101011
\newline
Ein Beispiel für das Zählen:
\newline \\
0 -> 1 -> 10 -> 11 -> 100 -> 101 -> 110 -> 111 -> ... 0111 1111 -> 1000 0000 -> 1000 0001
\newline \\
Die Kosten einer Transition von i -> i+1 ist definiert als die Anzahl der ’carries’, die bei der
Transition auftreten.
\newline
Notiz: ’carries’sind die Überträge, die beim (schriftlichen) Addieren (hier ’+1’) auftreten (können),
in Hardware-Addierwerken von CPUs sind sie wichtig für die notwendigen Zyklen der Operation.
\newline
Analysieren Sie die Laufzeit $T(n)$ des Algorithmus in Abhängigkeit von $n$.
\newline
\begin{enumerate}[(a)]
    \item Überlegen Sie wie teuer eine Operation (= ’transition’) sein kann!
    \newline
    Angenommen, i hat $k$ Bits:
    \newline
    In worst-case kann meistens $k$ Positionen ’carries’ auftreten wenn Position i -> i+1 übergeht. Damit beträgt die Kosten $O(nk)$.
    In best-case wird aber genau nur 1 Position ’carries’ auftreten somit ist die Kosten in $O(1)$.
    Dann kann es abgeleitet werden, dass $\sum kosten$ in Interval $(1, nk)$ liegt.
    Weil ’carries’ nicht bei jeder ’transition’ genau $k$ mal vorkommt und die Kosten von $n$ abhängig ist, betrachten wir die $\sum \frac{kosten}{n}$ als eine 
    unendliche Reihe, dann sollte diese Reihe konvergent sein.
    \newline
    Vermutung: Die Operation (= ’transition’) besitzt $O(n)$ Laufzeit-komplexität.
    \item  Beweisen oder widerlegen Sie $T(n)$ = $O(n)$!
    \newline
    Amortisierte Algorithmenanalyse:
    \newline
    Sei i = $n$ eine binäre String-Darstellung des Zählers und i hat $k$ Bits sodass bestrachten wir i als ein $A$ Array und $|A| = k$.
    \newline \\
    \begin{tabular}{||c|c|c|c|c||}
        \hline $n=$ & A[2] & A[1] & A[0] & ’carries’ \\
        \hline 0 & 0 & 0 & 0 & 0 \\
        \hline 1 & 0 & 0 & 1 & +1 \\
        \hline 10 & 0 & 1 & 0 & +2 \\
        \hline 11 & 0 & 1 & 1 & +1 \\
        \hline 100 & 1 & 0 & 0 & +3 \\
        \hline
        \end{tabular}
    \newline \\
    Durch amortisierte Analyse folgt, dass ’carries’
    \newline
    in A[0] in jeder ’transition’ vorkommt; \\
    in A[1] in jeder zwei ’transition’ vorkommt; \\
    in A[2] in jeder vier ’transition’ vorkommt; \\
    in A[3] in jeder acht ’transition’ vorkommt; \\
    in A[n] in jeder $2^{k-1}$ ’transition’ vorkommt; \\
    Damit gilt $T(n)$: 
    \begin{equation*}
        \begin{aligned}
            T(n) &= n + \lfloor n/2 \rfloor + \lfloor n/4 \rfloor + \lfloor n/8 \rfloor + \dots + \lfloor n/2^{k-1} \rfloor \\
                 &\leq \sum_{i=0}^{k-1} \frac{n}{2^i} \qquad (\text{geometrische Reihe: }\sum_{n=0}^{\infty}\frac{1}{2^n}) \\
                 &\leq 2n \\
                 &\Longrightarrow T(n) \in O(n)
        \end{aligned}
    \end{equation*}
    \item Was ist $T(I)$, wenn $I$ wie üblich die Instanzgrösse des BinaryCounter Problems ist?
    \newline
    Angenommen, $n$ sei die entsprechende binären Darstellung zu $I$ und $|n| = k$. Jede Bit der $n$ hat genau
    zwei Möglichkeiten: 0 und 1. Somit gilt: $2^k \geq I$ und $2^{k-1} \leq I$.
    \begin{equation}
        2^k \geq I \Longrightarrow 
        k \geq \log I \Longrightarrow 
    \end{equation}
    \begin{equation}
        2^{k-1} \leq I \Longrightarrow k-1 \leq \log I \Longrightarrow 
        k \leq \log I + 1
    \end{equation}
    Aus (1)und (2) folgt, dass $\log I \leq k \leq \log I + 1$. Somit gilt $T(I) \in O(\log I)$.
\end{enumerate}
\end{flushleft}
\end{document}