\documentclass{article}
\usepackage[a4paper,left=10mm,right=10mm,top=15mm,bottom=15mm]{geometry}
\usepackage{amssymb,amsthm,latexsym,amsfonts, amsmath, bm}
\usepackage{extarrows}
\usepackage{enumerate}
\usepackage{german}
\title{Übungen zur Algorithmischen Bioinformatik I\\
Blatt 6}
\author{Xiheng He }
\date{Mai 2021}
\linespread{1.5}
\begin{document}
\maketitle
\begin{flushleft}
\textbf{3. Aufgabe: Palindrom (10 Punkte)}
\newline
Ein Palindrom ist ein Wort $w = w_1w_2 \dots w_n$, das vorwärts gelesen gleich ist wie rückwärts gelesen, also $w = \stackrel{\leftarrow}{w}$ 
bzw. $\forall_{x=1,\dots,n} w_x = w_{n+1-x}$.
\begin{enumerate}[(a)]
    \item Gegeben sei ein Wort $w \in \{0,1\}^*$ auf dem Band einer 1-band deterministischen Turing Maschne (DTM). 
    \newline
    Definieren Sie formal eine DTM (d.h. $\varGamma$, $Q$ und $\delta $), welche die Palindrom-Eigenschaft für dieses
    Wort entscheidet. Beschreiben Sie kurz, was jede einzelne Regel leistet.
    \newline
    $DTM = (Q, \Sigma , \varGamma , \delta, q_0, q_{accept}, q_{reject});$ \\
    $\Sigma = \{0,1\}^*$; \\
    $\varGamma = \{0, 1, \varepsilon\}$; \\
    $Q = \{q_0, q_{accept}, q_{reject}, q_1^0, q_1^1, q_2^0, q_2^1, q_3\}$;
    \newline \\
    $\delta:=$
    \begin{tabular}{|| c | c | c | c ||}
        \hline $\delta$ & 0 & 1 & $\varepsilon$ \\
        \hline
        \hline $q_0$ & $(q_1^0, \varepsilon, R)$ & $(q_1^1, \varepsilon, R)$ & $q_{accept}$ \\
        \hline $q_1^0$ & $(q_1^0, 0, R)$ & $(q_1^0, 1, R)$ & $(q_2^0, \varepsilon, L)$ \\
        \hline $q_1^1$ & $(q_1^1, R, 0)$ & $(q_1^1, R, 1)$ & $(q_2^0, \varepsilon, L)$\\
        \hline $q_2^0$ & $(q_3, \varepsilon, L)$ & $q_{reject}$ & $q_{accept}$ \\
        \hline $q_2^1$ & $q_{reject}$ & $(q_3, \varepsilon, L)$ & $q_{accept}$ \\
        \hline $q_3$ & $(q_3, 0, L)$ & $(q_3, 1, L)$ & $(q_0, \varepsilon, R)$ \\
        \hline
    \end{tabular}
    \newline \\
    $q_0$: akzeptiert $\varepsilon$ oder fährt nach rechts zum weiteren Vergleich und ersetzen das aktuelle Zeichen mit $\varepsilon$. \\
    $q_1^0$ und $q_1^1$: markieren ein aktuelle Zeichen (0 oder 1) und fahren weiter nach rechts bis $\varepsilon$. \\
    $q_2^0$ und $q_2^1$: Vergleichen das letzte Zeichen vor $\varepsilon$ mit 0 oder 1 und gehen in $q_3$ über falls zwei 
    Zeichen übereinstimmen; ablehnen falls nicht übereinstimmen; akzeptieren falls das Zeichen $\varepsilon$ ist. \\
    $q_3$: fährt nach links bis $\varepsilon$. \\
    \item Analysieren Sie das Laufzeitverhalten dieser DTM.
    \newline
    Nach $\delta$ in DTM kann es leicht abgeleitet werden, dass es $n + 1 + n$ Schritte braucht um zwei Zeichen zu Vergleichen.
    \begin{equation}
        \begin{aligned}
            &(n + 1) + n + (n - 1) + (n - 2) + (n - 3) + (n - 4) + (n - 5) + \dots + 1 \\
            & = \sum_{i = 0}^{n+1} i \\
            & = \frac{(n + 2)(n + 1)}{2}
        \end{aligned}
    \end{equation}
    Aus (1) ist die Laufzeit dieser DTM in $O(n^2)$ 
    \item Betrachten Sie nun eine k-Band-DTM, d.h. neben dem Eingabeband sind k-1 weitere Arbeitsbänder vorhanden, die 
    wie das Eingabeband verwendet werden können (also auch einen eigenen Schreib-/Lesekopf besitzen). 
    Wieviele Bänder sind ausreichend, um das Problem effizienter als die 1-Band-DTM zu lösen? 
    Beschreiben Sie die Arbeitsweise dieser k-Band-DTM.
    \newline \\
    Um das Problem effizienter als die 1-Band-DTM zu lösen betrachten wir eine 2-Band-DTM, d.h. $k=2$, ein Band ist $w$ wie das Band in 1-Band-DTM und andere ist $\stackrel{\leftarrow}{w}$, 
    d.h. diese Band ist umgedreht sodass es rückwärts gelesen werden kann. Bewegen alle zwei Bänder bis $\varepsilon$ und vergleichen 
    jede Zeichen auf zwei Bänder das Lese-Schreibkopf aktuell zeigt. Falls zwei Zeichen inzwischen nicht übereinstimmen, 
    dann geht 2-Band-DTM in $q_{reject}$ und das Wort ist kein Palindrom. Falls keine Zeichen bis zum Ende nicht übereinstimmen, dann ist das Wort 
    ein Palindrom und akzeptiert 2-Band-DTM es.
    \item Analysieren Sie das Laufzeitverhalten der k-Band-DTM.
    \newline \\
    Zum Bestimmen ein Palindrom braucht 2-Band-DTM genau $2n$ mal Operationen (Band Bewegung 1 mal und Vergleich 1 mal) 
    wobei $n$ Länge des Wortes ist.
    Somit beträgt die Laufzeit der 2-Band-DTM in $O(n)$. 
\end{enumerate}
\end{flushleft}
\end{document}