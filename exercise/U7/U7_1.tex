\documentclass{article}
\usepackage[a4paper,left=15mm,right=15mm,top=15mm,bottom=15mm]{geometry}
\usepackage{amssymb,amsthm,latexsym,amsfonts, amsmath, bm}
\usepackage{extarrows}
\usepackage{enumerate}
\usepackage{german}
\title{Übungen zur Algorithmischen Bioinformatik I\\
Blatt 7}
\author{Xiheng He, Lisanne Friedrich}
\date{Juni 2021}
\linespread{2.0}
\begin{document}
\maketitle
\begin{flushleft}
\textbf{1. Aufgabe: Turing Maschine (10 Punkte)}
\newline
\textbf{Angabe:}
Definieren Sie (formal) eine Turing Maschine, die das Hamilton Circuit (HC) Problem polynomiell
auf das Traveling Salesman Problem (TSP) reduziert. Benutzen Sie die in der Vorlesung vorgestellte Reduktion und eine geeignete Repräsentation der HC und TSP Instanzen, so dass die Turing
Maschine möglichst einfach ist. Analysieren Sie die Laufzeit Ihrer Turing Maschine.\\
\textbf{Reduktion:}
Als Reduktion nehmen wir die auf Folie 118 beschriebene.\\
\textbf{Turing Maschine: }\\
Y,1 bedeuten es existiert eine Kante, N und 2 es existiert keine Kante\\
$ M=(Q,\Sigma ,\Gamma ,\delta ,q_0,\square,F)$\\
$ Q=\{v,e,accept\}$\\
$\Sigma =\{c,[0-9],/,Y,N\}$\\
$\Gamma =\{c,[0-9],/,Y,N\}$\\
$q_0 \in$ Q\\
F=\{accept\}\\
$\delta$: (v,c) $\rightarrow$ (v,c,R)\\
(v,Y) $\rightarrow$ (e,1,R)\\
(e,N) $\rightarrow$ (e,2,R)\\
(e,$\square$) $\rightarrow$ (accept,$\square$,N)\\
Die Turing Maschine geht solange nach Rechts bis sie ein Kantenymbol findet, mit dem sie überfüllen kann. So geht sie weiter vor, bis alle Kanten überfüllt sind.\\
Zum Schluss geht sie bis zum letzen Kantensymbol, auf das ein $\square$ folgt und accepted dann.
\newpage
\textbf{Laufzeitanalyse:} Die Laufzeit der Reduktion ist abhängig von der Inputlänge $n$. 
Sie liegt also in $\mathcal{O}(n)$. Der Graph kann bei $n$ Knoten maximal über $n-1$ Kanten verfügen. 
$\Rightarrow T(n)=n*(n-1)=n^2-1 \in O(n^2)$ \\
\textbf{Korrekheitsanalyse:} Es handelt sich um Elemtaroperationen, diese überschreiben von links nach rechts alles. Solange die Eingabe korrekt ist, ist der Algorithmus korrekt.
\end{flushleft}
\end{document}