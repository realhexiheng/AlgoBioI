\documentclass{article}
\usepackage[a4paper,left=10mm,right=10mm,top=15mm,bottom=15mm]{geometry}
\usepackage{amssymb,amsthm,latexsym,amsfonts, amsmath, bm}
\usepackage{extarrows}
\usepackage{enumerate}
\usepackage{german}
\title{Übungen zur Algorithmischen Bioinformatik I\\
Blatt 7}
\author{Xiheng He }
\date{Mai 2021}
\linespread{1.5}
\begin{document}
\maketitle
\begin{flushleft}
\textbf{3. Aufgabe: Reduktion: HP auf HC (10 Punkte)}
\newline
Ein Hamiltonscher Pfad (HP) ist ein Pfad, der jeden Knoten eines Graphen genau einmal besucht.
Reduzieren Sie das Entscheidungsproblem, ob es in einem Graphen einen Hamiltonschen Pfad gibt,
auf das Hamiltonscher Kreis (HC) Problem aus der Vorlesung.
\newline
Hinweis: Für eine polynomielle Reduktion muss neben der Korrektheit auch die polynomielle Laufzeit
nachgewiesen werden!
\newline \\
Sei zwei ungerichtete Graphen $G$ mit $G:= (V,E), V = \{v_1,\dots,v_n\}$ und $G'$ mit $G':= (V',E')$ .
Wir konstrurieren $G'$ durch:
\begin{itemize}
    \item Hinzufügen eines Vertex $v_{new}$ in $G$, sodass $V' = V \cup v_{new}$  
    \item Verbinden $v_{new}$ mit jedem Vertex $v \in V$ in $G$, sodass $E' = E \cup \{v_i, v_{new}|v_i \in V\}$
\end{itemize}  
Es ist leicht zu zeigen, dass die Laufzeit in $O(|V|)$ ist um $G'$ zu konstrurieren.
\newline
Daraus folgt, dass die Reduktion $f$ polynomielle Laufzeit besitzt, weil $T(f(x)) \in O(|V|) < O(n^k)$.
\newline
Anschließend zeigen wir: $\forall x \in \textbf{HP} \Leftrightarrow f(x) \in \textbf{HC}$
\begin{itemize}
    \item $\forall x \in \textbf{HP} \Longrightarrow f(x) \in \textbf{HC}$
    \newline
    Angenommen, dass \emph{Hamiltonscher Pfad} (HP) im $G$ existiert, dann gibt eine Permutation der knoten $\{v_{\pi(1)},v_{\pi(2)},
    v_{\pi(3)},\dots,v_{\pi(n)}|n = |V|\}$ im $G$.
    Somit existiert auch eine entsprechende Permutation der Knoten $\{v_{new}, v_{\pi(1)},v_{\pi(2)},
    v_{\pi(3)},\dots,v_{\pi(n)}, v_{new}\}$ im $G'$ für $V'$. Diese neue Permutation ist offensichtlich einen 
    \emph{Hamiltonscher Kreis} (HC) im $G'$ da alle knoten genau einmal besucht wurden und sich einen geschlossene Kreis 
    bilden. Im Gegenteil falls \emph{Hamiltonscher Pfad} (HP) im $G$ nicht existiert, dann gibt keine solche Permutation, 
    um einen \emph{Hamiltonscher Kreis} (HC) in $G'$ zu bilden.
    \item $\forall x, f(x) \in \textbf{HC} \Longrightarrow x \in \textbf{HP}$
    \newline
    Angenommen, dass einen \emph{Hamiltonscher Kreis} (HC) in $G'$ gibt, dann existiert eine Permutation
    $\{v_0, v_{\pi(1)},v_{\pi(2)},v_{\pi(3)},\dots,v_{\pi(n)}, v_0\}$. Damit bildet die neue Permutation $\{v_{\pi(1)},v_{\pi(2)},
    v_{\pi(3)},\dots,v_{\pi(n)}\}$ einen \emph{Hamiltonscher Pfad} (HP) im $G$ mit $\{v_{\pi(i)},v_{\pi(i+1)}|v_i \in V\} \in E$.
    Im Gegensatz wenn keiner \emph{Hamiltonscher Kreis} (HC) in $G'$ existiert, gibt es auch keine solche Permutation um den entsprechenden 
    \emph{Hamiltonscher Pfad} (HP) in $G$ zu bilden.
\end{itemize}
Somit reduzieren wir \emph{Hamiltonscher Pfad} (HP) auf \emph{Hamiltonscher Kreis} (HC).
\end{flushleft}
\end{document}