\documentclass{article}
\usepackage[a4paper,left=10mm,right=10mm,top=15mm,bottom=15mm]{geometry}
\usepackage{amssymb,amsthm,latexsym,amsfonts, amsmath, bm}
\usepackage{extarrows}
\usepackage{enumerate}
\usepackage{german}
\title{Übungen zur Algorithmischen Bioinformatik I\\
Blatt 7}
\author{Xiheng He, Lisanne Friedrich}
\date{Juni 2021}
\linespread{2.0}
\begin{document}
\maketitle
\begin{flushleft}
\textbf{2. Aufgabe: Orakel (10 Punkte)}
\newline
\textbf{a)}
\newline
\textbf{Angabe:}
Angenommen Sie haben ein Orakel, welches das Travelling Salesman Entscheidungsproblem in
konstanter Zeit für Sie löst (Entscheidungsproblem: Gibt es eine Tour, die nicht länger als $k$ ist?). 
Entwerfen Sie einen Algorithmus, der das Optimierungsproblem in polynomieller Zeit $(\mathcal{O}(n^p), p \geq 1)$ löst 
(Optimierungsproblem: Bestimmung der Tour mit der kürzesten Länge). Bestimmen Sie die asymptotische Laufzeit Ihres Algorithmus. 
Die Instanzgröße ist hierbei die Anzahl der Knoten $\vert V \vert$ .Die Kantengewichte des Graphen sind positive 
ganze Zahlen und hängen nicht von $\vert V \vert$ ab.\\
Sei ein Graph von G=(v,e) und die Kantengewichte von G durch d:$\mapsto \mathbb{N} $ gegeben. Das maximale Kantengewicht eines knoten v ist das maximale Geicht aller Kanten die v als Endknoten haben.
M ist die Menge aller maximalen Kantengewichte, daraus folgt, dass es sich bei $M_v$ um das maximale Kantengewichts eines Knoten v.\\
Die Strecke einer Lösung 
Max ist die Summe aller maximalen Kantengewichte. Jede Lösung der TSD Längen als Max gleich lang. Mit Hilfe einer binären Suche wird eiene Zahl l gesucht für die gilt: $\gamma(l)=1 und \gamma (l-1)=0$. Bei l handelt es sich um die Länge der längsten Lösung des TSP. Die binäre Suche ist abhängig von Max aber unabhängig von $\vert V \vert$. Durch die Entfernung aller Kanten e die die Bedinung $\gamma(l)=1$ in einem Graphen $G'=(V,E'), E'=E / \{e\}$ erfüllen, bleiben nur Kanten übrig, die zur Lösung des Optimierungsproblem gehören und alle kürzesten Strecke der Länge l gefunden.\\
\textbf{Laufzeitanalyse: }$\vert V \vert$=n\\
Ermitteln aller Kanten und maximaler Kantengewichte: $\mathcal{O}(\frac{n\cdot(n-1)}{2})=\mathcal{O}(n^2)$\\
Ermitteln von Max: $\mathcal{O}(n)$\\
Ermitteln von l (unabhängig von n): $\mathcal{O}(1)$\\
Entfernen überflüssiger Kanten: $\mathcal{O}(n^2)$\\
$\Longrightarrow$ Algoritmus läuft in $\mathcal{O}(n^2+n^2+n+1)=\mathcal{O}(n^2)$\\
Die Laufzeit ist polynomiel mit $n^k$, mit k=2;\\
\end{flushleft}
\end{document}
