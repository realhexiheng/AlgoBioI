\documentclass{article}
\usepackage[a4paper,left=15mm,right=15mm,top=15mm,bottom=15mm]{geometry}
\usepackage{amssymb,amsthm,latexsym,amsfonts, amsmath, bm}
\usepackage{extarrows}
\usepackage{enumerate}
\usepackage{tikz}
\usepackage{tikz-qtree}
\usepackage{german}
\title{Übungen zur Algorithmischen Bioinformatik I\\
Blatt 9}
\author{Xiheng He }
\date{Juni 2021}
\linespread{2.0}
\newtheorem{theorem}{Theorem}[section]
\begin{document}
\maketitle
\begin{flushleft}
\textbf{3. Aufgabe: PDP (10 Punkte)}
Aus der Vorlesung ist Ihnen das Partial Digest Problem (PDP) und ein backtracking Algorithmus
zu seiner Lösung bekannt. Gegeben sei die Multimenge
\newline
$A = \{1, 2, 4, 5, 6, 7, 8, 9, 11, 12, 13, 14, 15, 19, 20\}$ von $\binom{6}{2}$ paarweisen Distanzen.
Rekonstruieren Sie aus den gegebenen Distanzen die Lage der Restriktionsstellen unter Verwendung des in der Vorlesung
beschriebenen Verfahrens.
\newline
Geben Sie alle Schritte (Zwischenlösungen) sowie das endgültige Ergebnis an. 
\newline
\begin{enumerate}[(a)]
    \item $X = \{0,20\};A = \{1,2,4,5,6,7,8,9,11,12,13,14,15,19\};y = 19; S = (); \delta (19,X) = \{1,19\} \subseteq A$
    \newline \\
    \begin{tikzpicture}[scale=0.6]
        \coordinate (A) at (0, 0);
        \coordinate (B) at (19, 0);
        \coordinate (C) at (20, 0);
        \draw[-] (A)--(C);
        \draw (A) node [below]{0} -- ++(0, 3pt);
        \draw (B) node [below]{19} -- ++(0, 3pt);
        \draw (C) node [below]{20} -- ++(0, 3pt);
    \end{tikzpicture}
    \item $X = \{0,19,20\};A = \{2,4,5,6,7,8,9,11,12,13,14,15\};y = 15; S = (19); \delta (15,X) = \{4,5,15\} \subseteq A$
    \newline \\
    \begin{tikzpicture}[scale=0.6]
        \coordinate (A) at (0, 0);
        \coordinate (B) at (15, 0);
        \coordinate (C) at (19, 0);
        \coordinate (D) at (20, 0);
        \draw[-] (A)--(D);
        \draw (A) node [below]{0} -- ++(0, 3pt);
        \draw (B) node [below]{15} -- ++(0, 3pt);
        \draw (C) node [below]{19} -- ++(0, 3pt);
        \draw (D) node [below]{20} -- ++(0, 3pt);
    \end{tikzpicture}
    \item $X = \{0,15,19,20\};A = \{2,6,7,8,9,11,12,13,14\};y = 14; S = (15,19); \delta (14,X) = \{1,5,6,14\} \not \subseteq A$ \\
    $\delta (20 - 14,X) = \delta (6,X) = \{6,9,13,14\} \subseteq A$
    \newline \\
    \begin{tikzpicture}[scale=0.6]
        \coordinate (A) at (0, 0);
        \coordinate (B) at (6, 0);
        \coordinate (C) at (15, 0);
        \coordinate (D) at (19, 0);
        \coordinate (E) at (20, 0);
        \draw[-] (A)--(E);
        \draw (A) node [below]{0} -- ++(0, 3pt);
        \draw (B) node [below]{6} -- ++(0, 3pt);
        \draw (C) node [below]{15} -- ++(0, 3pt);
        \draw (D) node [below]{19} -- ++(0, 3pt);
        \draw (E) node [below]{20} -- ++(0, 3pt);
    \end{tikzpicture}  
    \item $X = \{0,6,15,19,20\};A = \{2,7,8,9,11,12\};y = 12; S = (6,15,19); \delta (12,X) = \{3,7,8,12\} \not \subseteq A$ \\
    $\delta (20 - 12,X) = \delta (8,X) = \{7,11,12,8\} \subseteq A$
    \newline \\
    \begin{tikzpicture}[scale=0.6]
        \coordinate (A) at (0, 0);
        \coordinate (B) at (6, 0);
        \coordinate (C) at (8, 0);
        \coordinate (D) at (15, 0);
        \coordinate (E) at (19, 0);
        \coordinate (F) at (20, 0);
        \draw[-] (A)--(F);
        \draw (A) node [below]{0} -- ++(0, 3pt);
        \draw (B) node [below]{6} -- ++(0, 3pt);
        \draw (C) node [below]{8} -- ++(0, 3pt);
        \draw (D) node [below]{15} -- ++(0, 3pt);
        \draw (E) node [below]{19} -- ++(0, 3pt);
        \draw (F) node [below]{20} -- ++(0, 3pt);
    \end{tikzpicture} 
\end{enumerate}
\newpage
\textbf{Suchbaum für PDP Instanz:}
\newline \\
\begin{center}
    \begin{tikzpicture}[scale=1,font=\small,
        every node/.style={black,thick,draw}
        edge from parent/.style={black,thick,draw},
        level 1/.style={sibling distance=8cm},
        level 2/.style={sibling distance=4cm}]
    \node {\{0, 20\}}
    child {node {\{0, 19, 20\}}
        child {node {\{0, 15, 19, 20\}}
        child {node {\{0, 6, 15, 19, 20\}}
        child {node {\{0, 6, 8, 15, 19, 20\}}}
        child {node {}}}
        child {node {}}}
        child {node {}}
        };
    \end{tikzpicture}
\end{center}
\end{flushleft}
\end{document}